% % % % % % % % % % % % % % % % % % % % % % % % % % % % % % % % % % % % % % % % 
% Formelsammlung von LaTeX4EI									
%
% @encode: 	UTF-8, tabwidth = 4, newline = LF
% @author:	Emanuel Regnath
% @date:		
%
% % % % % % % % % % % % % % % % % % % % % % % % % % % % % % % % % % % % % % % % 


%---------------------------------------%
%			P R E A M B L E				%
%~~~~~~~~~~~~~~~~~~~~~~~~~~~~~~~~~~~~~~~%

% Document Class ===============================================================
\documentclass[fs, footer]{latex4ei}



\newcommand{\sm}[1]{\langle#1\rangle}


% Dokumentbeginn
% ======================================================================
\begin{document}


% Aufteilung in Spalten
\begin{multicols}{4}
% -----------------------------------------------
% | 			R A U S C H E N			 		|
% ~~~~~~~~~~~~~~~~~~~~~~~~~~~~~~~~~~~~~~~~~~~~~~~
%===============================================================================================================================

% TITLE ========================================================================
\fstitle{Rauschen}


\emphbox{ Rauschen ist ein stochastisches, gleichanteilfreies Signal,\\ welches keiner Regelmäßigkeit folgt. }

	\section{Rauschen im Zeitbereich}

\sectionbox{
	Allgemeine rauschende Größe $A(t)$ mit \\
	Mittelwert bzw. Gleichanteil $\ol{A(t)} = \lim\limits_{T \ra \infty} \frac{1}{2T} \int\limits_{-T}^T A(t) \diff t$\\
	Rein rauschende Größe $a(t) = A(t) - \ol{A(t)}$ \qquad\qquad $\ol{a(t)} = 0$\\

	Für kleine $\tau$ lässt sich abschätzen in welchen Bereich das Signal verlaufen wird. (Max. Freq.)
}


\sectionbox{
	\subsection{Schwankungsquadrat (2. zentrales Moment, Varianz)}
	Ein Maß für die Rauschleistung ist die Varianz oder\\
	Schwankungsquadrat $\sigma^2 = \ol{a^2} = \ol{A^2} - \ol{A}^2$\\
	mit Gesamtleistung $\ol{A^2}$ und Gleichleistung $\ol{A}^2$\\ 
	\boxed{ \underset{\text{Gesamtleistung}}{\ol{A^2}} = \underset{\text{Gleichleistung}}{\ol{A}^2} + \underset{\text{Rauschleistung}}{\ol{a^2}} }\\
	Effektivwert (RMS) $\sigma = \sqrt{\ol{a^2}}$\\
	\\
	Höhere Momente: $\ol{A^n} = \int\limits_{-\infty}^\infty A^n \P(A) \diff A$\\
}


\sectionbox{	
	\subsection{Scharmittel}
	Bei $N$ gleichen Verstärker\\
	$\sm{A} = \lim\limits_{N \ra \infty} \frac{1}{N} \sum A$\\
	Falls $\sm{A} \equiv \ol{A}$, dann egodisches Rauschen\\
	Fast alle Rauschenarten in dieser VL sind stationäres Rauschen\\
}

	\subsection{Korrelation}
	Korrelationskoeffizient $c_{12} = \frac{\ol{a_1 a_2}}{\sqrt{\ol {a_1^2}} \sqrt{\ol {a_2^2}} } = \frac{\ol{a_1 a_2}}{\sigma_1 \sigma_2}$\\
	$c_{12} = 0$ notwendig aber nicht hinreichend für unkorrelierte Größen.\\
	


\sectionbox{
\subsection{Korrelation}
Ein Maß für die Ähnlichkeit zweier Signale $x(t), y(t)$ bei Verschiebung.\\
Korrelationskoeffizient $c_{xy} = \frac{E_{xy}}{\sqrt{E_x \cdot E_y}} = \frac{\varphi_{xy}(0)}{ \sqrt{\varphi_x(0) \cdot \varphi_y(0)}}$\\
\\
Es gilt: Korreliert $c = 1$, Orthogonal $\rho = 0$, Antipodisch $\rho = -1$\\
\\
\textbf{Kreuzkorrelationsfkt.} zwischen zueinander verschobenen Signalen:\\
	\emphbox{ $\displaystyle \varphi_{xy} (\tau) =  \varphi_{yx}(-\tau) = \int\limits_{-\infty}^{\infty} x(t) \cdot y(t+\tau) \diff t$ }
	Zusammenhang mit Faltung: $\varphi_{xy} (\tau) = x(-t) * y(t)|_{t = \tau}$\\
	\\
\textbf{Autokorrelationsfkt. AKF} ist Kreuzkorrelation mit sich selbst ($y = x$):\\
$\varphi_x (\tau) = \varphi_{xx}(\tau)$ \qquad Anwendung: Erkennen von Perioden\\ 
\\
\textbf{Energiebeziehung:} $E_{x,y} = \rho_{x,y} \sqrt{E_x E_y}$ mit \\
\textbf{Energie} $E_x = \int\limits_{- \infty}^{\infty} x(t)^2 \diff t = \int\limits_{- \infty}^{\infty} \Phi_x \diff f = \varphi_{xx}(0)$ \quad (endl. Sig.)\\
\textbf{Leistung} $P_x = \E[\X^2] = \frac{1}{2T} \int\limits_{-T}^{T} x(t)^2 \diff t$ \qquad\qquad (period. Sig.)\\
\textbf{Leistungsdichtespektrum} $\Phi_x(f)$ ist definiert als $\varphi_x \FT \Phi(f)$\\

Periodische Signale: $\overline \varphi_{xy}(\tau) = \frac{1}{2T} \int_{-T}^{T} x(t) y(t+\tau) \diff t$\\
\textbf{Stochastische Signale:} $\varphi_{\X\Y}(\tau) = \E[\X(t) \cdot \Y(t+\tau)]$\\
$\rho_{\X,\Y} = \frac{\Cov[\X \Y]}{\sigma_{\X} \sigma_{\Y}}$\\
$\int \limits_{-\infty}^{\infty} \PSD_{\X}(f) \diff f = \ACF_{\X}(0) = \Var[\X] + \E[\X]^2 = \sigma^2_{\X} + \mu^2_{\X}$
}


\sectionbox{
	\subsection{Grenzwerte}
	$\rho_a (0) = \ol{a^2} = \sigma^2$ \qquad $\rho_a(\infty) = 0$\\
	$\rho_A (\tau) = A(t) A(t+\tau) = \ol{A}^2 + \rho_a (\tau)$\\
	$\rho_A (0) = \ol{A}^2 + \ol{a^2} = \ol{A^2}$\\
	$\rho_A (\infty) = \ol{A}^2$\\
}


\sectionbox{
	\subsection{Folge identischer unabhängiger Impulse}
	Impulsefolge $A(t) = \sum\limits_i A_i(t)$ \qquad $A_i(t) = g_0 (t-t_i)$\\
	Campbellsches Theorem $\sigma^2 = \ol z \int\limits_{-\infty}^{+\infty} g^2_0(t) \diff t$\\
	$\ol z$ ist die Rate. Summe der Energie der Einzelimpulse ist die Energie der Fluktuation.\\
	Ankommende Impulse als Poisson-Prozess
}


	
	

% ======================================================================
\section{Arten von Rauschen}
% ======================================================================

	Weißes Rauschen: Konstante Frequenzverteilung.\\
	Rosa Rauschen: Fällt mit $\frac{1}{f}$ ab.

	Effektive Rauschspannung $U_{\ir eff} = \sqrt{\ol{U^2_1} + \ol{U^2_2}}$

\sectionbox{
	\subsection{Thermisches Rauschen}
	Durch die thermische Gitterschwingungen in einem Leiter erfolgt die Bewegung der Ladungsträger chaotisch.\\
	Effektivwert des Rauschens: $I_n = \sqrt{4 k_{\ir B} \cdot T \cdot R \cdot \Delta f}$\\
	Spektrale Dichtefunktion $W_i(f) = 2 e I_0$ \qquad für $f < \SI{e12}{\hertz}$\\
	Leistungsdichtespektrum $W$
	
	\begin{tabular}{ll}
	Widerstand & Leitwert\\ \mrule
	$\ol{u^2} = 4 k_{\ir B} \cdot T \cdot R \cdot \Delta f$ & $\ol{i^2} = 4 k_{\ir B} \cdot T \cdot G \cdot \Delta f$\\
	$W_u(f) = 4 k_{\ir B} \cdot T \cdot R$ & $W_i(f) = 4 k_{\ir B} \cdot T \cdot G$
	\end{tabular}

	Blindwiderstände ($C,L$) geben keine Rauschleistung ab!! \\
	$W_u(f) = 4k_{\ir B} T \Re{\cx Z(f)}$ \qquad $W_u(f) = 4k_{\ir B} T \Re{\cx Y(f)}$\\
	
	Beispiel $R \parallel C$: $W_u = \frac{4k_{\ir B}TR}{1+(2\pi f RC)^2}$\\
	$\ol{u_{c^2}} = \frac{k_{\ir B} T}{2}$ \qquad (Bandbegrenztes weißes Rauschen)\\
	
	Schwarzer Strahler: $k_{\ir B} T \ge h \cdot f$ Daraus folgt bei $T=\SI{300}{\kelvin}$ weißes Rauschen  bis $\SI{6}{\tera\hertz}$\\
	
		\subsubsection{Rauschleistung am Widerstand}
		Maximale Leistung $R = R_{\ir L}$\\
		$P_{\ir V} = P(R) = \frac{\ol{U^2}}{4 R} = k_{\ir B} T \Delta f$

		\subsubsection{Modell von Drude}
		Annahmen: 
		\begin{itemize}
			\item isotrope Geschwindigkeitsverteilung
			\item Freie Flugzeit zwischen Stößen $\tau_{\ir C} = \const$
			\item Energie ist $\frac{k_{\ir B} T}{2}$
			\item unabhängige Bewegung
		\end{itemize}

		$i_q = \frac{e}{l} \cdot v_{xq}$\\
		$W_0(f) = 4 \int\limits_0^{\tau_C} \ol{i^2} \left( \frac{\tau_C - \tau}{\tau_C} \right) \diff \tau = 2 \ol{i^2} \tau_C$\\
		$\ol{i^2} = e^2 \frac{n \cdot A}{l} \cdot \frac{k_{\ir B} T}{m}$\\
		Leitwert: $W_0(f) = 4 k_{\ir B} T G$
}

\sectionbox{	
	\subsection{Schrotrauschen}
	Ursache: Quantisierung der Ladung.\\
	Schrotrauschen tritt bei Stromfluss $I_0$ über eine Potentialbarriere auf.\\
	Tunnelndes Teilchen, Poisson-Prozess, Rate bekannt, Zeitpunkte zufällig.\\
	Effektivwert des Rauschens: $I_n = \sqrt{\ol{i^2}} = \sqrt{2 e \cdot I_0 \cdot \Delta f}$\\
	Spektrale Dichtefunktion 	
	$W_{\ir Schrot}(f) = 2 \frac{I_0}{e} \abs{\mathcal F_g(f)}^2$\\
	$W_0 = 2 e I_0$ \qquad (mit Impulsform $g$)
}	

\sectionbox{	
	\subsection{Generations-Rekombinations-Rauschen im TDGGW}
	$\tau_0$ mittlere Lebensdauer im Zustand 0. Mit WSL $\P_0 = \frac{\tau_0}{\tau_0 + \tau_1}$\\
	$\tau_1$ mittlere Lebensdauer im Zustand 1. Mit WSL $\P_1 = \frac{\tau_1}{\tau_0 + \tau_1}$\\
	$\diff \tau$ muss so klein sein, dass nur ein Übergang stattfindet.\\
	Übergänge $\P_{10} + \P_{11} = 1$\\
	$\P_{11} (\tau + \diff \tau) = \P_{11}(\tau) \left[1- \frac{\diff \tau}{\tau_1}\right] + \P_{10}(\tau) \frac{\diff \tau}{\tau_0}$\\
	DGL: $\frac{\P_{11}(\tau + \diff \tau) - \P_{11}(\tau)}{\diff \tau} = \frac{\diff \P_{11}}{\diff \tau} = \frac{1}{\tau_0} - \P_{11} (\tau) \cdot \frac{1}{\tau_\rho}$\\
	Lösung: $\rho_A(\tau) = \P_1 (1-\P_1) \exp \left( -\frac{\tau}{\tau_\rho} \right) + \P_1^2$\\
	Maximales Rauschen falls Besetzung der Niveaus $\P_1 (1-\P_1) = 0.25$
}

Onsager Prinzip: Betrachtet man die Relaxation einer Störung einer bestimmten Größe, so ist der zeitlicher Ablauf dieser Relaxation gleich der AKF des Rauschens dieser Größe

\sectionbox{
	\subsection{Rauschen stromdurchflossener Widerstände}
	Leitfähigkeit änderrt sich durch G/R
	$W_I(f) = \frac{I_0^2}{N_0^2} W_N(f)$\\
	
	
}


\sectionbox{
	\subsection{1/f Rauschen}
	Generell: Überlagerung von Bandbegrenzten, gleichmäßigen Rauschen.\\
	System mit begenzten Energiegehalt in welchem nichtlineare Prozesse für eine Verteilung der Energie auf einem breiten Frequenzbereich sorgen. Daraus folgt 1/f-Rauschen.
	Wärmeleitung, RC-Rauschen, Diffusionsvorgänge, verteilte Netze\\
	1/f-Rauschen ist energetisch sehr günstig.
	
}


\sectionbox{
	\subsection{Zeitskaleninvarianz}
	Leistung in einem Frequenzintervall zwischen $f_1$ und $f_2$:\\
	$P = c \ln\left(\frac{f_2}{f_1}\right)$ \qquad allg. Konstante $c$\\
	Leistung konstant im relativen Frequenzintervall.
}


\sectionbox{
	\subsection{Geometrische Abhängigkeit des Rauschens}
	Bei einem homogenen Volumen $V$ ist die relative Schwankung proportional zu $\frac{1}{V}$\\
	Bsp. Spannungseinprägung: $\frac{W_i}{I_0^2} = \frac{1}{V} \frac{W_{\vec j}}{\norm{\vec j}^2}$
	Mit $\Delta U = 0$ und $\Delta E = 0$\\
	\\
	Bei einer homogenen Fläche $A$ sind die Schwankungen proportional zu $\frac{1}{A}$

}


\sectionbox{
	\subsection{Hooge-Modell (Mathematisches Modell)}
	Anzahl der freien Ladungsträger $N$ schwankt\\
	$\left. \frac{\ol{\Delta I^2}}{I_0^2} \right|_{\Delta U=0} = \left. \frac{\ol{\Delta U^2}}{U_0^2} \right|_{\Delta I=0} = \frac{\ol{\Delta R^2}}{R_0^2} = \frac{\ol{\Delta N^2}}{\ol {N}^2}$

	Überlagerung mehrerer Zeitkonstanten.
	$W_{N} \propto \ol{\Delta N^2}$ \qquad $W_{N} \propto \frac{1}{f}$\\
}

\sectionbox{
	\subsection{McWhorter-Modell (Physikalisches Modell)}


}

\section{Übertragung von Rauschen über elektrische Netzwerke}
% ======================================================================
\sectionbox{
	\subsection{Übertragungsfunktion}
	Achtung: Bei Rauschen betrachten wir Leistungsspektren $W \propto \ol{a^2}$\\
	Phaseninformationene gehen verloren.\\
	$W_a(f) = \abs{G(f)}^2 \cdot W_e(f)$ \\
	Bei Filterung von weißem Rauschen sieht man die Übertragungsfunktion ($\dirac \FT 1$)\\
	
	Äquivalente Rauschbandbreite: $B_{\ir eq} = \frac{\int\limits_0^{\infty} \abs{G(f)}^2 \diff f}{\abs{G(f)}^2_{\max}}$\\
	$\rho_a(0) = \ol{a^2} = W_{\ir e} \cdot \abs{G(f)}^2_{\max} \cdot B_{\ir eq}$\\ 

}


\sectionbox{
	\subsection{Impedanzfeldmethode}
	Liefert Aussagen über interne Rauschgrößen durch die Schwankungsgrößen an den Klemmen.\\
	
	Modellannahmen: Eindimensionale Betrachtung (in $x$ Richtung) räumlich verteilter, linear verknüpfter Rauschquellen (nicht homogen und statistisch unabhängig)\\


	Impedanzfeld: $i(x) \ra \diff u_a(x)$ \quad wegen $\diff u = i(x) \cdot \cx z(x)$\\
	
	Für untereinander unkorrelierte Rauschgeneratoren gilt:\\
	$W_i(x,x',f) = w_i(x,f) \cdot \dirac(x-x')$\\
	$W_u(f) = \int\limits_0^L w_i(x,f) \cdot \abs{z(x)}^2 \diff x$\\
	   
		\subsubsection{Beispiel thermisches Rauschen eines Widerstandes}
		$z(x) = \frac{1}{\sigma A}$ \qquad $w_i(x) = 4 k_{\ir B} T \sigma A$\\
		$W_u(f) = \int\limits_0^L 4 k_{\ir B} T \frac{\sigma A}{\sigma^2 A^2} \diff x= 4 k_{\ir B} T \frac{L}{\sigma A} = 4 k_{\ir B} T R$

}

\section{Kenngrößen rauschender Virpole}



\sectionbox{
	\subsection{Rauschzahl von Zweitoren}
	Rauschzahl $F = \frac{\SNR_1}{SNR_2} = \frac{N_2}{G \cdot N_1} = \ge 1$\\

	

}


\section{Diodenrauschquellen}
	Eine Diode rauscht ohne angelegte Spannung wie der kleinste Ersatzleitwert ohne Spannung.




	\subsection{Diodenrauschquellen}
	$i_{\ir eff} = \sqrt{2 e I_0 \Delta f}$
	$F = 1 + 20 \frac{I_0 R_G}{Volt}$


	\subsection{Vergleich von Schrot- und thermisches Rauschen}
	$\ol{i^2} = K 4 k T g_0 \Delta f$



	\subsection{Mindestmesszeit}
	Effektivwert: $\lim\limits_{T \ra \infty} \int_0^T ... \diff t$\\
	Gleichanteil: $W_{=} = 8B^2W_0^2 \dirac(f)$\\
	Wechselanteil: $W_\sim = 8 W_0^2 \dirac(B-f)$\\
	Mittelwert: $\ol A = 2B W_0$\\
	Schwankungsquadrat: $\ol{a^2} = \frac{2 B W_0^2}{\tau_M}$\\
	


	







Anwendungen logarithmischer Plots:\\
Semi-log für $y = e^{ax}$. liefert gerade\\
Doppellog für $\frac{1}{1+\left(\frac{x}{x_0}\right)^\beta}$. Bodediagram.
Gerade bedeutet?



	\begin{tabular}{lll}
	Analytisch & Stochastisch\\
	Mittelwert & Erwartungswert\\
	Schwankungsquadrat & Varianz\\
	Abweichung & Standardabweichung\\
	\end{tabular}






	Elektronentemperatur:
	$w_i = 4 k_{\ir B} T_e G = 4 k_{\ir B} T_e \frac{A e n}{L} \mu$









% ======================================================================
\section{Praktikum}
% ======================================================================
Op-Amps als invertierender Verstärker




% Ende der Spalten
\end{multicols}

% Dokumentende
% ======================================================================
\end{document}
